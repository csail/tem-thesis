\section{TEM Host Driver}\label{arch:driver}

The TEM is an add-on to a computer, and therefore requires driver software
(also called host software) on the host computer. The host software design
should be easily derived from the design of the TEM it interfaces with. For the
sake of completeness, this section discusses the main issues encountered during
the development of the prototype driver.

At a minimum, the TEM host software should provide the following services:
\begin{itemize}
  \item {certification;} The TEM is useless unless the driver can produce an
  Endorsement Certificate that can be used by the TEM's owner to assure
  $3^\textrm{rd}$ parties that the TEM's PubEK can be trusted.
  \item {execution;} The TEM provides trusted execution by carrying out
  instructions contained in bound SECpacks. The host software must be capable
  of loading a SECpack into the TEM, and retrieveing the execution result from
  the TEM.
  \item {key deletion;} As discussed in section \ref{arch:key_store}, the TEM
  owner must be able to delete keys from the cryptographic engine's store, in
  order to prevent denial of service attacks.
  \item {persistent store deletion;} The persistent store architecture may
  allow the owner to delete associations (section \ref{arch:pstore_freshness}),
  in order to avoid denial of service attacks from malicious SEClosures. If that
  is the case, the driver should implement the operation. Furthermore, the
  driver should keep track of the closure that created each association, and
  allow the user to make informed decisions when deleting associations.
\end{itemize}

