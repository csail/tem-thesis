\section{Communication Interface}\label{arch:interface}
The main component of the TEM's communication interface is a transceiver for
the communication channel between the TEM and its owner. This channel can be an
external bus such as the Universal Serial Bus \cite{specification2000r} if
the TEM is a physically distinct computer add-on, or an internal bus such as
the PCI local bus \cite{specification1998r} or the LPC (low pin-count) bus
\cite{intel2002lpc}, if the TEM is included on the system's motherboard.

If the communication channel between the TEM and its owner is insecure, the
communication interface abstracts this problem away from the rest of the TEM
modules, by establishing a secure session. The mechanism for establishing the
secure session is inspired by SSL \cite{freier1996ssl}, but simplified by the
lack of protocol negotiation. A rough outline of the required steps is:
\begin{enumerate}
  \item the TEM sends its Endorsement Certificate,
  \item the owner validates the Endorsement Certificate and is assured that the
  Public Endorsement Key (PubEK) belongs to a TEM,
  \item the owner generates a symmetric encryption key\footnote{if the laws prohibit symmetric
  encryption, an asymmetric encryption key can be used instead} to be used as a
  session key,
  \item the owner encrypts the session key with the TEM's PubEK and sends it to
  the TEM,
  \item the TEM uses its Private Endorsement Key (PrivEK) to decrypt the
  session key, and
  \item the TEM and the owner use the session key to communicate securely.
\end{enumerate}

