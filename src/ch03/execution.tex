\section{Execution Engine}\label{arch:execution}
The execution engine processes SECpacks synchronously, according to
the following simple loop:
\begin{enumerate}
  \item load a SECpack provided by the communication interface,
  \item carry out the computation expressed in the SECpack,
  \item provide the computation result to the communication interface, and
  \item release resources and prepare for the next SECpack.
\end{enumerate}

\subsection{The TEM Virtual Machine}\label{arch:vm}
The computation inside a SECpack is expressed as microinstructions
for a stack-based virtual machine (VM). The virtual machine stores code and data
in a single, flat memory space (described in detail later). The entire VM
interpreter state consists of the following registers:
\begin{enumerate}
  \item {IP (the instruction pointer)} is the memory address of the instruction
  that will be executed next, and
  \item {SP (the stack pointer)} is the memory
  address of the top of the VM's stack.
\end{enumerate}

Instructions are encoded as a 1-byte operation code (opcode), optionally
followed by immediate data. The size of the immediate data following an opcode
can be determined by looking at the opcode, to make code analysis easy.

The stack has a conventional design: it consists of fixed-size entries, which
are the size of the machine word. Unless otherwise specified, instructions read
their inputs by popping entries off the stack, and push their results onto the
stack.

\subsubsection{Standard Operations}
The virtual machine has standard instructions for performing arithmetic
operation, stack manipulation, and execution flow control.

\subsubsection{Single Memory Space}
The TEM's execution environment provides a single RAM-backed memory space that
contains the closure's executable instructions, values of local variables and de-facto
immutable non-local variables, and the virtual machine's stack.

Although the TEM's VM has a stack-based instruction set, closures have
unrestricted access to the memory space. This offers maximum flexibility (e.g.,
self-modifying code) that closure compilers can use to squeeze one last drop
of performance out of the execution engine.

\subsubsection{The Output Buffer}
The output buffer is an append-only memory zone, and its role is to make
building secure closures easier. If a closure terminates successfully by
executing \texttt{halt}, the contents of the output buffer is returned as the
result of the execution. In case an exception occurs during execution, nothing
is returned. Therefore, the output buffer is intended to help developers build
secure closures, by demanding that they make an active effort to report a piece
of data as a result.

\subsubsection{Persistent Storage Interface}
The contents of mutable non-local variables is stored in the TEM's persistent
store (sections \ref{concepts:pstore} and \ref{arch:pstore}) between
executions. Addresses are stored in memory space, and values are transferred
between the memory space and the persistent store. The instructions for working
with the persistent store closely mirror the interface presented in
section \ref{arch:pstore}.

\subsubsection{Cryptographic Engine Interface}
The execution engine offers interfaces to all the functions of the
cryptographic accelerator described in section \ref{arch:crypto}.

A closure is automatically authorized to use all the encryption keys that it
creates. The SEClosure can use keys that are already loaded in the cryptographic
engine, once it demonstrates knowledge of their authorization secret. The
execution engine enforces these restrictions by keeping track of the keys that
the closure is authorized to use. If a closure attempts to use a key before
gaining authorization, its execution is aborted. The execution engine
also keeps a list of the temporary keys produced by a SEClosure, and removes them at the end of the closure's execution.

Keys in the cryptographic engine's store are identified by the number of the
slot storing them. Unless otherwise specified, the cryptographic instructions
read/write key identifiers from/to the VM stack.

\subsection{Design Philosophy}\label{arch:vm_philosophy}
This section describes the issues encountered in the design the TEM's
execution engine, and explains the reasoning behind the trade-offs that were
made.

\subsubsection{Motivation for Synchronous Execution}
The execution engine design completely discards any possibility of concurrent
execution. The factors driving this decision are:
\begin{itemize}
  \item Concurrent programming is notoriously hard to get right, so
  applications demanding provable security tend to forgo the speed benefit
  anyway.
  \item At the time of this writing, all existing low-cost secure chips contain
  a single general-purpose execution core, so provisions for concurrent
  executions would not be readily applicable.
  \item A multi-core TEM can be modeled as multiple execution engines that
  share a persistent store. The consistency of the persistent store is ensured 
  by treating each SECpack execution as a transaction. This design is much
  easier to implement correctly, as it relies on heavily-studied techniques
  from the database world.
\end{itemize}

\subsubsection{Motivation for the Single Memory Space}
The clear separation between the RAM-backed memory space and the consistent
store provides optimum performance in low-resource environments. Most
operations reference RAM memory, which yields speed and increases EEPROM
lifetime.

Since the single memory space contains a closure's code, allowing unrestricted
access to that memory space makes static verification techniques (used in the
Java Virtual Machine) impossible to apply. This is not an issue for the TEM, as
the target hardware does not have the resources needed for static verification,
so dynamic verification is unaviodable.

The TEM would not benefit from a more complex memory scheme. Segmented memory
would not be useful, since the execution engine is synchronous, so code sharing
cannot occur. Paged memory is also not interesting, because the TEM does not
have multiple code privilege levels, or virtual memory.

\subsubsection{Motivation for a Virtual Machine}\label{arch:vm_motivation}
Using a virtual machine carries a price in performance, but makes the the executable code in a SECpack universal. Having SECpack executable code
target specific hardware introduces complexity (multiple target platforms) for
the closure compiler, and blocks SECpacks from being migrated among TEMs.

The TEM's virtual machine plays a bigger role in execution speed than commodity
VMs that are targeted towards desktop computers (e.g., the Java Virtual Machine
\cite{lindholm1999jvm}, and ECMA Common Intermediate Language
\cite{standard2001cli}) because the TEM is not powerful enough to perform
Just-In-Time Compilation (JIT). Thus, on the TEM, even performance-critical
code is running under the VM interpreter, whereas desktop virtual machines
can translate ``hot'' code to the hardware's native instruction set.

On the other hand, the performance degradation introduced by the virtual machine
has no significant impact on most TEM applications. Assuming a reasonable VM
implementation, a single asymmetric key encryption operation dwarfs the cost of
interpreting thousands of VM instructions. Public-key encryption is invoked
for every bound SECpack, because a part of the SECpack must be decrypted with
the TEM's Private Endorsement Key.

\subsubsection{Motivation for a Stack-Based Instruction
Set}\label{arch:stack_motivation}
Having established that the TEM will use a virtual machine makes the
decision of using a stack-based instruction set relatively straight-forward.

Stack-based instruction sets are easy to generate from ASTs (Abstract Syntax
Trees), and are a decent intermediate representation for code analysis
algorithms. Stack-based instruction sets also yield the smallest possible code.
If JIT (Just-in Time Compiling) becomes a possibility, simple register
allocators can turn stack-based code into native code with very good
performance. 

Using a register-based instruction set doesn't make much sense for an
interpreted virtual machine. The real processor registers are likely to be used
up by the VM interpreter, so virtual registers would end up being simulated by
RAM, which results in the same speeds seen by stack-based languages, at the
cost of a more complex implementation.

Stack-based instruction sets are used in recent VMs for both medium-level
languages (e.g., the Java Virtual Machine \cite{lindholm1999jvm}) and high-level
languages (the Ruby 1.9 VM \cite{sasada2005yya}).

\subsubsection{Considerations in the Design of the Instruction
Set}\label{arch:vm_instructions}
The standard instructions are heavily inspired by the Java Virtual Machine
\cite{lindholm1999jvm}. The TEM-specific instructions (cryptography and
persistent store) have been developed trying to apply the same principles.

The instruction set tries to strike a balance between enabling small SECpacks
and keeping the VM interpreter simple. For example, most instructions operating
on memory blocks have two variants. The fixed block variant (instructions ending
in \texttt{fxb}) receives the information about the blocks (address, and
optionally length) as immediate data, which optimizes SECpack size. The
variable block variant (instructions ending in \texttt{vb}) pops the block
information off the stack, for maximum efficiency when working with
variable-length memory structures. Exceptions were made for instructions that
would not occur often in a SECpack (e.g., \texttt{rnd}), so the space savings
do not warrant the extra complexity in the interpreter.

The design of the TEM-specific instructions was biased by my laziness which
pushed me to keep the prototype implementation simple.

The instruction set aims for consistency with respect to mnemonics and
order of parameters. A few examples:
\begin{itemize}
  \item \texttt{ld} (load) instructions push data on the stack, and \texttt{st}
  (store) instructions write data to the memory space,
  \item \texttt{fxb} (fixed-block) instructions receive their parameters in the
  same order that they should be pushed on the stack for the corresponding
  \texttt{vb} (variable-block) instructions, and
  \item instructions that receive the same parameters (e.g., \texttt{mcfxb} and
  \texttt{mcpyfxb}) have the same parameter order.
\end{itemize}
Aside from making the VM easier to understand, consistency can be leveraged
to optimize the interpreter code.

\subsection{The SECpack Loader}\label{arch:secpack_loader}
A SECpack consists of a snapshot of the initial state of the virtual machine's
memory space, together with a header containing a magic value, the initial
values for the VM's instruction pointer and stack pointer, and data needed to
decrypt a bound SECpack. This makes virtual machine setup trivial, given an
unbound SECpack.

A bound SECpack (section \ref{concepts:secpack_binding}) requires that the
loader decrypts the private information $\mathcal{P}$ and verifies the
integrity of the shared information $\mathcal{S}$. The process for doing this
is straightforward, and implies reversing the steps of the binding process
described in section \ref{concepts:secpack_binding}. The SECpack header
contains the sizes of the $\mathcal{S}$ and $\mathcal{P}$ areas, and an unbound
SECpack is indicated by an empty $\mathcal{P}$ area.

When the SECpack loader is invoked, it is given the SECpack and the number of
a slot in the cryptographic engine's key store. The key in that slot is used to
decrypt the SECpack. The loader must reject bound SECpacks that fail the
integrity check $\mathcal{H} = h(\mathcal{S} || \mathcal{P})$.

The ability of using arbitrary keys (as opposed to PrivEK alone) for decrypting
SECpacks allows for speed optimizations and extensions to the TEM's chain of
trust. For example, an often-used set of SECpacks can be bound to a TEM using a
symmetric key instead of PubEK, if the key is transmitted securely to the
TEM by encrypting it with PubEK. This can significantly decrease execution time
by avoiding an asymmetric-key decryption operation every time the SECpacks are
loaded.

The TEM trust chain is not compromised by allowing arbitrary keys to be
used as SECpack decryption keys, because of the integrity check that occurs after
SECpack decryption, and because of the magic value in the header. 

\subsection{The TEM Instruction Set}
For the sake of completeness, this section briefly describes the instructions
that make up the VM Instruction Set. Although the instruction set is usually
considered an implementation detail, the VM's instruction set is necessarily a
part of the TEM's architecture. All TEMs, regardless of the hardware
implementation, will have to implement the same instruction set. The
instruction set presented here is not completely optimized, but it is a good
starting point, as it is working well in the prototype TEM implementation. 

The instructions are a straightforward application of the design principles set
forth in section \ref{arch:vm_philosophy}, combined with bits shamelessly
stolen from the other architectures I have worked with. Most readers can safely
skip this section.

\subsubsection{Standard Operations}
The standard instructions, classified by their purpose, are:
\begin{itemize}
  \item arithmetic operations; \texttt{add}, \texttt{sub} (subtract),
  \texttt{mul} (multiply), \texttt{div} (division quotient), \texttt{mod}
  (division reminder) perform respective arithmetic operation on the top two
  numbers in the stack, and push the result on the stack.
  \item flow control; \texttt{halt} marks the successful completion of a
  closure's execution. \texttt{jmp} changes the value of the IP register
  (``jumps'') unconditionally, as opposed to the conditional jump instructions that only change IP if the value at the top of the stack meets the condition in the instruction name. The conditional jumps are (unsurprisingly) \texttt{jz} (jump if zero), \texttt{jnz} (jump if not zero),
  \texttt{ja} (jump if above zero), \texttt{jb} (jump if below zero),
  \texttt{jae} (jump if above or equal to zero), and \texttt{jbe} (jump if
  below or equal to zero). The new IP value is encoded as immediate data for
  each of the flow control instructions.
  \item stack manipulation; \texttt{ldbc} (sign-extend and load/push a 1-byte
  constant), \texttt{ldwc} (load/push a 1-word constant), \texttt{pop} (pop one
  item), \texttt{popn} (pop $N$ items), \texttt{dupn} (duplicate the top $N$
  items), \texttt{flipn} (reverse the order of the top $N$ items) are rather
  straightforward. Constants are encoded as immediate data. $N$ is considered a
  constant encoded as 1-byte unsigned number.
\end{itemize}

\subsubsection{Single Memory Space}
The following instructions for interfacing with the memory space take one
immediate value, which is the address that they read from / write to.
\begin{itemize}
  \item \texttt{ldb} (load byte): sign-extends and loads/pushes a byte in the
  memory space
  \item \texttt{ldw} (load word): loads/pushes a word beginning at the
  given address in the memory space
  \item \texttt{stb} (store byte): pops the top word off the stack and stores 
  its least-significant byte in the memory space
  \item \texttt{stw} (store word): pops the top word off the stack and stores
  it beginning at the given address in the memory space
  \item \texttt{mcfxb} (memory-copy, fixed block): copies a memory block
  (contiguous sequence of bytes) to another memory location; the operation
  parameters (address and length of the source block, address where the block
  will be copied) are encoded as word-sized immediate values
  \item \texttt{mcvb} (memory-copy, variable block): analogous to
  \texttt{mcfxb} that pops the operation parameters off the stack
  \item \texttt{mcmpfxb}, \texttt{mcmpvb} (memory-compare, fixed /
  variable blocks): lexicographically compare the contents of two memory blocks
  and store the comparison result on the stack; the buffers are identified
  using the same parameters as in \texttt{mcfxb}, and respectively \texttt{mcvb}
\end{itemize}

The instructions \texttt{ldbv} (load byte from variable address), \texttt{ldwv}
(load word from variable address), \texttt{stbv} (store byte at variable
address), and \texttt{stwv} (store word at variable address) behave similarly
to the instructions explained above, but they use the stack to read the
memory address that they will operate on. These instructions are useful when
dealing with variable-size data structures.

\subsubsection{The Output Buffer}
The instructions interfacing with the output buffer are:
\begin{itemize}
  \item \texttt{outnew} (new output buffer): creates the output buffer for a
  SEC; reads the word from the top of the stack as an upper limit of the number
  of bytes that will be written to the output buffer (an exception will be
  generated if the SEC exceeds the limit)
  \item \texttt{outb}, \texttt {outw} (output byte / word): same behavior as
  \texttt{stb}, \texttt{stw} except they target the output buffer
  \item \texttt{outfxb}, \texttt{outvb} (output fixed / variable block): same
  behavior as \texttt{mcfxb}, \texttt{mcvb}, except the destination is the
  output buffer
  \item \texttt{outvlb} (output variable-length block): a compromise between
  \texttt{outfxb} and \texttt{outvb} -- the address of the source memory buffer
  is fixed (encoded as immediate data), but the length is obtained from the
  stack
\end{itemize}

As a further optimization on SECpack size, a destination address equal to the
maximum word size (i.e., \texttt{0xFFFF} on 16-bit processors) is interpreted
as ``the destination is the output buffer''.

\subsubsection{Persistent Storage Interface}
The instructions for interfacing with the persistent store are:
\begin{itemize}
  \item \texttt{pswrfxb}, \texttt{pswrvb} (persistent store write using
  fixed/variable blocks): write a value from the memory space into the
  persistent store. Their parameters are the addresses of the memory blocks 
  containing the persistent store address and the value to be written. The
  instructions map to the \texttt{write} method presented in \ref{arch:pstore}.
  \texttt{pswrvb} has variable parameters (coming from the stack), whereas
  \texttt{pswrfxb} uses fixed parameters (encoded as immediate values).
  \item \texttt{psupfxb}, \texttt{psupvb} (persistent store update using
  fixed/variable blocks): similar to \texttt{pswrfxb}, \texttt{pswrvb}, but abort
  execution if the persistent store had no association for the given address (the
  underlying \texttt{write} would have returned \textsc{created})
  \item \texttt{psrdfxb}, \texttt{psrdvb} (persistent store read using
  fixed/variable blocks): read a value from the persistent store into the
  memory space. The parameters mirror \texttt{pswrfxb}, and respectively
  \texttt{pswrvb}. An execution is generated if the persistent store does not
  have a value associated with the given address.
  \item \texttt{pshk} (persistent store has key): looks up an address in the
  persistent store. The persistent store address is stored in a variable block
  (its location in memory space is read from the stack). The result is
  \texttt{1} if the persistent store contains an association for the given
  address, and \texttt{0} otherwise. The result is pushed on the stack.
  \item \texttt{psrm} (persistent store remove): removes a value from
  the persistent store, using the semantics of \texttt{remove} in
  \ref{arch:pstore}. The only parameter is the address of memory block
  containing the peristent store address. The parameter is read from the stack.
\end{itemize}

\subsubsection{Cryptographic Engine Interface}
The following instructions manage the encryption keys for the algorithms 
covered in sections \ref{arch:crypto_asymmetric} and
\ref{arch:symmetric_crypto}:
\begin{itemize}
  \item \texttt{genk} (generate key): generates a symmetric or asymmetric
  encryption key. The desired key type is encoded as a 1-byte immediate value.
  Upon successful completion, the instruction will push on the stack the
  slot/slots holding the generated key (an asymmetric key requires two slots).
  \item \texttt{relk} (release key): unloades a key from the crypto store.
  \item \texttt{ldkl} (load key length): pushes on the stack an upper bound for
  the length (in bytes) required to hold the serialized version (via
  \texttt{stk}) of a key.
  \item \texttt{rdk} (read key): creates a crypto store key by reading
  a serialized version (via \texttt{stk}) of the key from a memory block. The
  address of the memory block is popped from the stack. Upon success, pushes the
  number of the slot holding the new key on the stack.
  \item \texttt{stk} (store key): writes a serialized version of a key into the
  memory space. The address of the memory block and the slot number are read
  from the stack.
  \item \texttt{authk} (authorize key): pops off the stack a slot number and
  the address of a memory block containing an authorization secret. If the slot
  contains a temporary key, it is made persistent by attaching the authorization
  secret. Otherwise, the authorization secret is presented to gain access to
  the key. An execution exception occurs if the presented secret does not match
  the authorization secret associated with the key.
\end{itemize}

The encryption keys can be used to:
\begin{itemize}
  \item encrypt and decrypt data, via \texttt{kefxb}, \texttt{kevb},
  \texttt{kdfxb} and \texttt{kdvb} (keyed encrypt/decrypt using fixed/variable
  memory blocks), and
  \item produce and verify signatures, via \texttt{ksfxb}, \texttt{ksvb},
  \texttt{kvsfxb} and \texttt{kvsvb} (keyed sign/verify signature using
  fixed/variable memory blocks).     
\end{itemize}
The instructions above take the following parameters: the slot number of the
key to be used (always read from the stack), the address and length of the
block of data to be used as input, and the address where the output shall be
written. The last three parameters can be fixed (encoded as immediate values) or
variable (read from the stack).

The hashing function (section \ref{arch:crypto_hash}) in the cryptographic
engine is accessed by the instructions \texttt{mdfxb}, and \texttt{mdvb}
(message-digest using fixed/variable blocks). The instructions resemble those
for copying memory blocks (\texttt{mcpyfxb}, and \texttt{mcpyvb}) but, instead
of writing a copy of the source block, they write a cryptographic hash (also
known as a message digest) of the source block.

The random number generator (section \ref{arch:crypto_rng}) is invoked by the
instruction \texttt{rnd} (random), which writes random data to the memory
space. \texttt{rnd}'s two parameters are read from the stack, and they indicate
the desired number of random bytes, and the memory address where the random
bytes will be written.

Encryption keys are stored by the cryptographic engine. A closure's memory
space may contain serialized encryption keys, but these keys have to be loaded
into the cryptographic engine before they can be used in cryptographic
operations. If a TEM can generate encryption keys, the VM provides instructions
for serializing the keys into the running closure's memory space. 
