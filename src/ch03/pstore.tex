\section{Persistent Store}\label{arch:pstore}
The persistent store is an associative memory backing the values of
mutable non-local variables for all the closures executing on the TEM. Section
\ref{concepts:pstore} argues that the most appropriate design is to identify
variables by addresses that are as large as cryptographic hashes, and use the
knowledge of a value's address as proof of authorization to access that
variable.

The conceptual presentation of the persistent store leaves out the following
issues:
\begin{itemize}
  \item the operations supported by the persistent store,
  \item the nature of the values in the store,
  \item the method for assigning an address to a variable, and
  \item bridging the gap between the possibility of executing a large number of
  SECpacks, and the small amount of non-volatile memory in the low-cost secure
  chips that the TEM targets.
\end{itemize}
The omission is deliberate, as the answers to these questions are engineering
trade-offs, and would have detracted from the understanding of the concept. This
section provides answers that are appropriate in the context of the TEM
architecture.

\subsection{Persistent Store Operations}
In the absence of a contrary reason, the persistent store provides the standard
associative memory operations:
\begin{itemize}
  \item \texttt{read(address)} returns the value stored at \texttt{address}, or
  a special \textsc{nothing} value if the persistent store does not have an
  association for the given address.
  \item \texttt{write(value, address)} writes \texttt{value} at
  \texttt{address}. \texttt{write} overwrites the old value stored at the given
  address, or creates a new association for \texttt{address}, and returns
  \textsc{created} or \textsc{updated} to reflect what happened. The return
  value is not relevant for a general understanding of the TEM archiecture, but
  it is needed in section \ref{arch:pstore_external}.
  \item \texttt{remove(address)} removes the value associated with
  \texttt{address}, if such an association existed.
\end{itemize}

\subsection{Persistent Store Values}
All the values in the TEM's persistent store are the size of the hashes
produced by the TEM's cryptographic engine (section \ref{arch:crypto_hash}).

Using fixed-size variables is essential to avoid complexity in the persistent
store implementation. Furthermore, an implementation that provides
variable-size values will end up using most (if not all) of the saved memory on
the bookkeeping required to make variable lookup fast.

The reasoning behind the chosen size requires some understanding of the TEM's
execution engine, described in section \ref{arch:execution}. The unique memory
space and the stack-based instruction set suggest a single attractive
alternative to the choices of value size: the size of a machine word. This
choice would make programming a bit more convenient, because it would be
possible to read a persistent store value directly onto the stack, or write
calculation results directly from the stack to the persistent store.

However, word-sized values  would cause memory waste, because a persistent
store value would be much smaller\footnote{The chips suitable for the TEM at
the time of this writing have a word size of 16 bits, whereas a SHA-1 hash
takes 160 bits.} than its address, so most of the memory would be spent on
addresses.

Furthermore, larger persistent store values can translate into speed benefits.
Closure compilers can bundle multiple variable values into the same persistent
store entry, in order to optimize speed. This will always yield benefits,
because reading/writing from/to the persistent store implies NVRAM, and
therefore is bound to be slower than multiplexing values in a closure's
RAM-backed memory space.

In conclusion, making persistent store values approximately the same size
as addresses makes sense for the TEM. The slight increase in complexity for the
closure compiler is rewarded by increased SECpack execution speed, and
better memory utilization.

\subsection{Persistent Store Address Allocation}\label{arch:ps_alloc}
The size of persistent store addresses must be at least as large as a symmetric
encryption key (section \ref{concepts:pstore}). This allows the following
allocation strategy: randomly choosing a persistent store address allocates that
address on all the TEMs in the world. Address allocation is used to mean that
no other closure will use the same address for a different variable. Address
allocation does not imply memory allocation on any TEM.

This strategy has the advantages that a variable has the same address on all
TEMs, and that address allocation can be done off-line. On the other hand, the
method's robustness is not immediately obvious, and guaranteeing integrity
against replay attacks is non-trivial. An argument for the method's robustness
is provided below. The discussion on preventing replay attacks is complex
enough that it has been assigned to its own section \ref{arch:pstore_freshness}.

\subsubsection{Robustness Argument}
Allocating an address uses the same mechanism as generating a symmetric
encryption key. Addresses are large enough that the chance of a conflict is not
bigger than the chance of generating an encryption key that belongs to someone
else. Most of the world's economy relies on symmetric encryption, as banks
communicate with each other using SSL \cite{freier1996ssl} or TLS
\cite{dierks1999tpv}, which in turn use symmetric encryption.

An argument similar to that in section \ref{concepts:pstore_auth} can be
used to show that the probability of having an attacker break the TEM trust
chain is much greater than the probability of having two SECpacks on the same
TEM that use the same address for different variables.

In closing it is worth noting that increasingly many systems rely on UUIDs
(Universally Unique Identifiers) \cite{leach2005ruu} for what is
essentially off-line address allocation. Using a hardware random number
generator to allocate a persistent store address is stronger than
the pseudo-random UUID allocation algorithm proposed in \cite{leach2005ruu}.
It is assumed that a SEClosure author has access to a good hardware RNG.

\subsection{Guaranteeing the Freshness in the Persistent
Store}\label{arch:pstore_freshness}
Guaranteeing freshness when the persistent store addresses are allocated 
offline requires some thought. It is non-trivial for a SEClosure to distinguish
between the case when its mutable non-local variables have never been used on a
TEM (and thus should be assumed to contain default values), and the case when
its variables have been assigned values, but the corresponding associations
have been \texttt{remove}d from the persistent store. The possibility of
confusing the two cases can be exploited by replay attacks.

\subsubsection{Preventing Replay Attacks: the Easy Way}
The straightforward method of eliminating the ambiguity is to specify
that once a persistent store association is created, it persists for the
lifetime of the TEM. In other words, the operation \texttt{remove} is forbidden.

When adopting this solution, the following concerns need to be addressed:
\begin{itemize}
  \item the persistent store grows proportionally to the number of
  non-local mutable variables used by SECpacks throughout the life of the TEM,
  and
  \item malicious SECpacks can initiate a denial of service attack by filling
  up the persistent store.
\end{itemize}

However, this solution is easier to understand and implement, so it may be more
desirable in some situations. All off the issues above can be worked around.

The persistent store entries can be stored in untrusted memory (as explained in
section \ref{arch:pstore_external}), and the capacities of both hard disks and
flash memories has been increasing at a rate that exceeds Moore's law. Denial of
service attacks can be prevented by placing a hard limit on the number of new
persistent store variables can be used by a SECpack.

\subsubsection{Preventing Replay Attacks: the Painful but Efficient Way}
The method of preventing replay attacks explained above is easy to understand,
but has a potential drawback in that all the non-local mutable variables
used on a TEM must persist forever, even if the SECpacks referencing them
will never be executed again. This section describes a more complex method for
preventing replay attacks that allows unused values to be \texttt{remove}d from
the TEM. The presentation requires a significantly deeper understanding of the
TEM as a whole, and is not as self-contained as the previous section.

The problem of preventing replay attacks can be reduced to avoiding the replay
of the initialization of the persistent store variables. This is because normal
SECpacks assume that the relevant mutable non-local variables have been
initialized prior to execution, so persistent store \texttt{read}s will never
return \textsc{nothing}. If this assumption is broken, a SECpack aborts its
execution and does not return any information  o the TEM owner. 

The only computation that is vulnerable to replay attacks is now the
initialization of persistent store variables. At-most-once semantics can be
provided for initializations by linking them to a monotonic counter inside the
TEM. This approach requires that a single value, the monotonic counter, has to
persis throughout the lifetime of the TEM. The rest of the section explains
the idea in more detail.

Let an \textbf{object}\footnote{Object-Oriented Programming is used to simplify
the presentation. However, the mechanism presented here can be easily
adapted to other programing paradigms.} be a group of SECpacks that use the
same mutable non-local variables. For convenience, an object's \textbf{fields}
shall be all the mutable non-local variables used by the SECpacks in the
object. Using the bank account example in section \ref{concepts:closures}, an
individual bank account is an object, and it consists of the SECpacks labeled
\texttt{withdraw}, \texttt{deposit}, \texttt{balance}, and \texttt{number}. The
bank account has one field, the variable \texttt{balance} (\texttt{number} is
optimized away by the closure compiler because it is de-facto immutable).

To reduce complexity, the lifetime of all the fields of an object are managed
at once, and management follows the same principles as constructors and
destructors (also named finalizers). Namely, an object is \textbf{constructed}
on a TEM by assigning initial values to all its fields. An object is
\textbf{destroyed} by \texttt{remove}-ing the values of its fields from the
persistent store. The SECpacks in an object will abort execution if any of
the fields they reference do not have a value in the persistent store,
because that means the TEM owner destroyed the object prematurely.

\subsubsection{Preventing Replay Attacks on Object Construction}
The solution is completed by ensuring that an object is constructed at most
once. A simple mechansim that achieves this goal is described below.

The mechanism uses a single monotonic counter, $\mathcal{M}_C$. The monotonic
counter's value is stored in the persistent store, at an address known only to
privileged SECpacks (described in section \ref{arch:key_store}). 

Constructing an object is achieved series of assignments to persistent store
addresses. The object's owner can construct a list of (address, value) tuples
that completely capture this information. This list is referred to as the
\textbf{constructor table}. 

Yu follows the following steps to produce an object that Mii can use on his
TEM:
\begin{enumerate}
	\item Mii runs a privileged SECpack that returns the value of
	$\mathcal{M}_C$, signed with the TEM's Private Endorsement Key, $r_v =
	\mathcal{M_C} || Enc_\textrm{PrivEK}(h(\mathcal{M}_C))$.
	\item Mii gives Yu $r_v$, the result of reading the
	counter, together with the TEM's Endorsement Certificate.
	\item Yu verifies ECert and the signature, to make sure that
	the Public Endorsement Key in the certificate can be trusted, and that
	the value of $\mathcal{M}_C$ is authentic.
	\item Yu binds the object's SECpacks to the TEM using PubEK,
	and also encrypts the \textbf{constructor data}, which is the constructor table
	together with the value of $\mathcal{M}_C$ that was verified in the previous step. The results produced
	at this step are given to Mii.
	\item Mii runs a privileged SECpack that decrypts the constructor data using
	PrivEK, and verifies that the value of $\mathcal{M}_C$ matches the value in
	the constructor data. If the verification is completed, then the SECpack
	carries out the \texttt{write}s described in the constructor table, and
	increments $\mathcal{M}_C$.
\end{enumerate}

It is easy to see that the process above does not allow an object
to be constructed twice -- exactly one object is constructed for a certain
value of $\mathcal{M}_C$. Guaranteeing that Mii cannot replay the constructor
assignments translates into guaranteeing freshness for the fields of an
object, which are, by definition, all the values that SECpacks will read
from the persistent store.

\subsection{Secure External Memory: The World in a
Nutshell}\label{arch:pstore_external}
Section \ref{concepts:pstore} explains that the persistent store must protect
its contents from attempts of bypassing the associative memory interface.
This leads to the straightforward implementation of storing all the
associations in non-volatile memory shielded by the TEM's tamper-resistant
envelope. Unfortunately, secure NVRAM is expensive, so it does not come in
abundance on the low-cost chips that the TEM targets. This section analyzes the
possibility of using untrusted memory outside the TEM's secure envelope, which
is significantly less expensive.

The TEM uses the approach to external secure memory introduced by the AEGIS
\cite{suh2003aat} secure processor, which leverages Merkle trees
\cite{merkle1980ppk}. The persistent store's external memory design adapts the
maintainance algorithms to the TEM scenario (a low-cost chip attached to a
powerful, but untrusted processor), and modifies the design slightly to obtain a
hash tree whose height is logarithmically proportional to the number of
associations inside the TEM, and not to the size of the addressing space.

Like AEGIS \cite{suh2003aat}, the persistent store relies on building a tree,
where the leaves store the actual associations, and internal nodes store a
cryptographic hash of their children\footnote{Optimized designs that allow
internal nodes to store associations together with the hash of their children are
reasonably straightforward.}. The tree's root must be stored in NVRAM inside the
TEM's secure envelope, but all the other nodes can be stored in untrusted memory.

In order to ensure the integrity and confidentiality of the data, neither the
persistent store addresses nor the values can be stored ``in the clear'' in
untrusted memory. A symmetric encryption key\footnote{The symmetric
key never leaves the TEM, and can be generated cheaply using a PUF
\cite{gassend2002spr}. }\footnote{If the law does not allow symmetric encryption, an asymmetric key is used instead. Note that the persistent store becomes painfully inefficient in this case.} that is also stored inside the TEM's NVRAM is used to encrypt the associations. The
two parts of an association must be encrypted individually, so that the TEM can
later ask for an association by its encrypted address. This external
representation is used for computing the hashes in the internal nodes, so the
contents of the internal nodes are not confidential.

The TEM's host (with a powerful but untrusted processor) is responsible for
maintaining the tree structure. When a persistent store \texttt{read} is
issued, the TEM communicates the encrypted persistent store address to the
host. The host responds with the encrypted value associated with the
address, together with a proof of correctness consisting of the contents of all
the intermediary nodes. A \texttt{write} is handled similarly, except that the
correctness proof also describes the updates that must be performed to the
tree.

\subsubsection{Amnesia and Replay Attacks}
The tree used for external memory storage is said to exhibit
\textbf{amnesia} if it is possible for the TEM's host, who is managing the tree,
to ``forget'' about a persistent store association. The host would tell the
TEM that an entry does not exist, when in fact it does. Assuming the persistent
store implementation mistakenly believes the owner, the net result would be
that a \texttt{read} could return \textsc{nothing} when it should return a
value, and a \texttt{write} could return \textsc{created} instead of
\textsc{updated}.

An external memory tree exhibiting amnesia is \textbf{acceptable} if the
persistent store design, as a whole, can be used to detect the wrong answers
that can be originated by amnesia and report that the external tree has been
compromised.

Amnesia is important for the persistent store design. External
memory trees that are guaranteed to either provide the correct answer or
exhibit amnesia have a much simpler structure than trees that must provde the
correct answer, without the possibility of amnesia. If amnesia is not
acceptable, then the external tree sturcture must allow the TEM's host to
construct a prof that a tree contains no association for a given address. On
the other hand, if amnesia is allowed, then the host computer can structure the
tree as it wishes, and it does not need to provide a proof when claiming that
the tree does not contain any association for a certain address.

A necessary, but not sufficient, condition for amnesia to be acceptable is that
the correct return value of a \texttt{write} (\textsc{created} or
\textsc{updated}) is known in advance. In other words, whenever a variable is
assigned a value, the persistent store should know whether the store already
contains a corresponding association or not. Otherwise, the host can trick the
TEM into creating two leaves for the same association, and later lead the TEM
to the leaf corresponding to a stale value.

The method of guaranteeing persistent store freshness (section
\ref{arch:pstore_freshness}) determines if amnesia is acceptable. For example,
the straigthforward method described at the beginning of section
\ref{arch:pstore_freshness} requires the guarantee that the persistent store
will not exhibit amnesia, because its correctness relies on receiving the
correct answer from \texttt{read}, and amnesia means that a \texttt{read} may
return \textsc{nothing} when it shouldn't.

On the other hand, the more complicated method described in the same section
works correctly even when faced with amnesia, because it must work correctly
even when the TEM owner \texttt{remove}s associations (intuitively, amnesia
resembles spurious \texttt{remove}s.) The complicated method works correctly
becauseit is known that all the SECpacks of an object will issue only
\texttt{writes}s that return \textsc{updated}, and only \texttt{reads} that do
not return \textsc{nothing}. Also, all the \texttt{write}s in a constructor
are guaranteed to return \textsc{created} under normal use, but this condition
does not need to be verified to insure the correctness of the persistent store
operations.

\subsubsection{Preventing Amnesia}
If the persistent store is not allowed to exhibit amnesia, the host must be
able to prove that an association does not exist in the tree. In order for the
proof to have an efficient encoding (i.e., not enumerate all the leaves in the
tree), there must be a single possible path from the root to a leaf containing
a certain address.

The fixed tree in AEGIS \cite{suh2003aat} trivially meets the requirement
above, but is not ideally suited for a sparse addressing space, because its
height is proportional to the number of bits in an address. The issue is
exacerbated by the pressure of the low-memory TEM environment on the
tree's branching factor. For example, assuming SHA1 is used for
hashing, a branching factor of 1,000 would require 20KB of RAM for storing a
single internal node. 20KB is a luxury in the low-cost chips that the TEM is
targeting. On the other hand, using a really small branching factor increases
the tree depth and lowers performance.

A binary search tree is an attractive alternative, especially because the tree
\emph{does not have to be balanced}. Intuitively, it is safe to assume that the
tree will perform according to the average-case analysis, because the tree keys
are encrypted addresses, so they will ``look'' random. Formally, a misbehaving
SEClosure cannot cause the persistent store to produce a worst-case tree structure, because it cannot guess the addresses that would encrypt to keys that would lead to the worst-case tree structures.

Implementing the binary search trees requires that internal nodes would have to
be augmented to also store addresses, which is straightforward. An optimized
implementation will tweak the tree's branching factor to achieve a good
compromise between the size of an internal node and the tree's height.
