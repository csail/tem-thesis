\chapter{TEM Concepts}\label{concepts}

This chapter introduces the concepts used in the Trusted Execution Module. It
defines ambiguous or obscure terms, and explains the big-picture ideas
behind the TEM.

The TEM is a very small TCB (Trusted Computing Base) which can be enclosed in a
tamper-resilient envelope at commodity prices. The platform provides the
guarantees associated with trusted execution, even if the user requiring
trusted execution is not the TEM's owner.

The TEM's execution primitive is the closure. Section \ref{concepts:closures}
explains the benefits and research behind the decision.

The TEM provides the possibility of trusted execution on computers outside
one's ownership. Section \ref{concepts:trusted_execution} analyzes the
definition of trusted execution and explains the family of situations in which
the TEM can add value.

The root of trust in the TEM is an Endorsement Certificate produced by a
TEM's manufacturer, asserting that a public key (the Public Endorsement Key)
corresponds to a private key that is only known to unique TEM. Section
\ref{concepts:trust_chain} explains the chain of trust that leads to this
assertion.

Closures are transmitted to the TEM in a binary format that is convenient for
the TEM to process. Section \ref{concepts:secure_closures} begins with
by classifying the pieces of information inside a closure according to the
required guarantees. The bulk of the section is dedicated to introducing a
cryptography-based process that guarantees the confidentiality and integrity of
a closure's contents, as it is transmitted via unsafe channels to the TEM.

Closures can use non-local mutable variables, whose values must persist
across closure executions. The TEM stores all these values in an associative
memory. Addresses are the same size as encryption keys, so the knowledge of a
variable's address serves as proof of authorization to access the variable's
value. This rather unconventional design is covered in section
\ref{concepts:pstore}, where I argue for its robustness and minimality.

% Finally, section \ref{concepts:threat_model} explains the threat model under
% which the TEM functions.

