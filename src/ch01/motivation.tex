\section{Motivation: the Need for Trusted Computing}\label{intro:motivate}

Research on secure systems under the standard assumptions (no computer in the
system can be trusted) is hitting a hard ceiling: there is only so much we can do
without a trusted party in the system. The following cases illustrate this
point:

\begin{enumerate}
  \item The SUNDR paper \cite{li2004sud} arguments that fork consistency is
  the best guarantee a secure system can provide in the absence of an on-line trusted
  party. Fork consistency means that users are protected from an un-trusted
  server returning arbitrary data instead of their files, but they are not
  protected from a server that will return old versions of their files.
  
  \item In \cite{castro1pbf}, Castro and Liskov argue that a replicated
  state machine (the standard architecture for implementing fault-tolerant services) needs 3N+1
  replicas to survive N byzantine failures. Having trusted parties assist the
  consensus agreement protocol would reduce the number of replicas to 2N+1
  (since replicas need to vote on the correct answer). This would greatly
  reduce both message size and the number of messages needed to achieve N-fault
  tolerance.
     
	\item Peer-to-peer systems like Bittorrent \cite{cohen2003ibr} and Chord
	\cite{stoica2001csp} eliminate the single-point of failure and scalability concerns of central servers. However, the
	absence of trusted hosts renders peer to peer architectures unusable for general
	applications. The most notable example is MMO (Massive Multiplayer Online) games,
	which would benefit greatly from this technology, but need to withstand
	sophisticated attacks from players that want to gain unfair advantages.
	
	\item Mobile payment systems are gaining traction in the UK
	\cite{hickins2007mpsgt}, and are well-established methods of payment in South Korea and Japan
	\cite{bradford:clm}. However, since phones are un-trusted computers,
	transactions need to happen online, which raises anonymity concerns, and dooms mobile payments to a huge barrier to adoption
	experienced by any application that needs cooperation from cell-phone service
	providers.
\end{enumerate}

The situations presented above are all real problems whose resolution can
dramatically impact tomorrow�s technology landscape.
