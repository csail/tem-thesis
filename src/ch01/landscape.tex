\section{Landscape: Related Trusted Computing Work}\label{intro:landscape}

\subsection{Trusted Modules}\label{intro:trusted_modules}
Secure platforms have been in demand since the advent of computers,
especially by government intelligence agencies and by the financial industry.

Early solutions for secure platforms were supplied, most notably by IBM, as
tamper-resistant assemblies that can operate either autonomously or as
coprocessors for high-end systems. The latest incarnation of these systems is
the IBM 4764 co-processor \cite{arnold2004ipn}, which is only available for IBM
servers under custom contracts, which is pretty suggestive of the price range for the system.

The TPM (and its successor, the TEM) will not be in a position to
replace cryptographic processors anytime soon, as they focus on providing a
trusted platform at a low cost, which comes at a detriment to performance.

\subsection{Smart Cards}
Smart cards \cite{hendry2001scs} are secure platforms embedded in thumb-sized
chips. For handling convenience, the chips are usually embedded in plastic
sheets that have the same dimensions as credit cards. The same chips, without
the plastic sheets, are used as Subscriber Identity Modules (SIMs) in GSM
cellphones. Smart cards have become pervasive, by offering a secure platform at
a low cost.

The ISO 7816 standard regulates the low-level aspects of
smart cards \cite{husemann2001ssc}, namely the physical shape and disposition of
contact points, voltages accepted by the contact points, and the link-layer and network-layer
communication protocols between the smart card and its host device.

Platforms such as MultOS \cite{maosco:m} and JavaCard
\cite{microsystems2003jcp} provide a common infrastructure for speeding up
application development, and allow multiple applications from different vendors
to coexist securely. However, both of these platforms build monolithic
applications that are contained and executed completely on the
smartcard. Given the limited resources available in a smart card chip, the
approach above places a very low ceiling on the complexity of the applications
that can be developed.

\subsection{Secure Processors}
Secure processors represent a different approach to trusted platforms. A secure
processor costs less than a trusted module because the secure envelope only
contains the logic found inside CPUs. The AEGIS \cite{suh2003aat} design
provides a cost-effective method for implementing a secure processor.

Secure processors are much more powerful than the chips found inside
smart cards, and would be able to power applications that are significantly more
complex.


\subsection{The Trusted Platform Module (TPM)}\label{intro:tpm}
The TPM is the first security-related computer component that has gained mass
adoption, and is now included in laptop computers from major manufacturers such
as Apple and IBM. The important lessons learned from the TPM's strategy are:

\begin {itemize}
  \item The module specification should focus on the operations it must perform,
  and on the security requirements for the platform, without dabbling in actual
  chip design. This allows a variety of implementations coming from different
  vendors.
  \item The hardware required to build the module must be so cheap that its
  price is insignificant relative to the price of the untrusted platform it is
  attached to. Lack of high financial risk encourages manufacturers to adopt
  the technology.
  \item The specification should not use algorithms or concepts covered by
  export control or technology patents. This way, vendors can design, produce
  and sell modules anywhere in the world. Furthermore, since algorithms that are
  not covered by export control can be incorporated into a universal
  specification, this makes the platform more attractive for application
  writers.
\end{itemize}

\subsubsection{Limitations}
While having the merits of removing a lot of obstacles corresponding to
political and business practices in the computer manufacturing industry, the
solution proposed by the Trusted Computing Group is lacking from a technical
point of view. The TPM is a fixed-function unit, which means it defines a
limited set of entities (such as shielded locations holding a cryptographic key
or hash), as well as a closed set of operations that can be performed with the
primitives (such as using a key to unwrap another key or to sign a piece of
data). The TCG followed this avenue because it entailed simpler correctness
proofs and promised to allow really cheap implementations. However, the
fixed-function approach proved to be a poor match for the use cases envisioned
by the TCG, which lead to an explosion in the complexity of the TPM
specification. In response to a complex specification, vendors chose to use
reasonably sophisticated secure chips borrowed from the smart-card industry.

Furthermore, the vision for the TPM states that that its main goal is to attest
that the computer it is bound to is running a TCB (trusted computing base, as
defined in \cite{latham1985ddt} and \cite{lampson:ads}). The TCB notion
encompasses all software that, once successfully attacked, may impact the
correctness of computations executed by a program on the computer. The TPM
design includes the software in the TCB, so a trusted platform needs to run a
secure boot loader, operating system, and drivers. This is impossible to
achieve in practice for the following two reasons:

\begin{enumerate}
  \item The operating systems used in production (Windows, Linux, Mac OSX) have
  huge amounts of code running with administrative privileges, for performance
  reasons. It is impractical to analyze and certify such a large codebase,
  especially given the frequent stream of updates these operating systems
  expect. As a representative example, the Common Criteria Security Evaluation
  of Windows 2000 \cite{microsoft2002eal}, \cite{shapiro2003uwe} was completed
  in October 2002, which was more than one year after the following operating system version was released.
  \item System drivers are a part of the TCB, even on systems that run driver
  software in user mode, like MINIX 3 \cite{herder2006msp}. This is
  because a driver communicates with a hardware device which is connected to
  the system bus, and therefore has full read/write access to main memory via
  DMA transfers. This means that a security certification (e.g., the Common
  Criteria mentioned above) necessarily includes a hardware platform
  specification. Most systems are not willing to sacrifice agility for
  security, so large TCBs have proven impractical.
\end{enumerate}

Last but not least, the Achilles' foot of the TPM architecture is the bond
between the secure chip that is TPM and its host computer. The nature of the
bond implementation ultimately determines the security of the entire system,
because an attacker that compromises the bond can break the attestation system.
A perfectly secure bond ultimately amounts to enclosing everything connected
to the main bus in a secure envelope, which yields the expensive systems
described in section \ref{intro:trusted_modules}. The specification of the TPM
for PC systems claims that using the LPC (low pin-count) bus
\cite{intel2002lpc} as the bond is a good compromise between security and cost.
However, version 1.1 of the TPM specification has been broken by a trivial
one-wire attack \cite{lawson2007tha} on the LPC bus, and version 1.2 still
leaves room for a relatively simple attack \cite{lawson2007tha2}.

The TPM is still useful in the absence of trusted software, as shown by works
like \cite{sarmenta2006vmc}. However, the lack of general-purpose
computation places very narrow bounds on the applications of the TPM. In
practice, the chip is most often used to implement a secure key store to be
used in multifactor authentication, as illustrated by \cite{yoderpkcs11}.

\subsection{The Need for a TEM}
The Trusted Computing research group at MIT, led by Professor Srinivas Devadas,
has researched the applications afforded by the TPM, and understood its
limitations. The group submitted and received funds for a NSF grant proposal
containing the idea of a Trusted Execution Module that would be similar to the TPM, but provide
execution capabilites.

The proposal of adding execution capabilities to the TPM mentioned above was
the seed for this work. My thesis provides the result of exploring the idea
mentioned above. In the process of turning the idea into a concrete
implementation, the unnecessarily complex aspects of the TPM's design have been
discarded, and replaced with new mechanisms that support an elegant execution
model.
