\section{Thesis Outline}\label{intro:outline}
The rest of this thesis is structured as follows.

This chapter presented the context required to understand the
Trusted Execution Module, and makes an argument for the value provided by the
TEM. The facts here are not required for understanding the TEM, but it will
greatly enhance the reader's understanding of the motivation behind the design
decisions.

Chapter \ref{concepts} defines the concepts that form the basis of the TEM
architecture. Ambiguous or obscure notions are refined into clear definitions,
which are used to explain the big ideas behind the TEM.

Chapter \ref{arch} covers the architecture of the TEM. The chapter visits each
component in a TEM, providing a thorough presentation of the structures and
processes involved at that component, interwoven with an analysis that sheds
light on the reasoning behind the design decisions.

Chapter \ref{impl} presents a prototype implementation of the TEM architecture
proposed in chapter \ref{arch}. The implementation consists of a full stack,
starting from firmware for a commodity secure chip, and going all the way to
demonstration software that uses the TEM.
