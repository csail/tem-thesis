\chapter{Introduction}\label{intro}

The Trusted Execution Module (TEM) is a Trusted Computing Base (TCB) designed
for the low-resource environments of inexpensive commercially-available secure
chips. The TEM can execute small computations expressed as compiled closures.
The TEM guarantees the confidentiality and integrity of both the computation
process, and the information it consumes and produces. The TEM's guarantees
hold if the compiled closure author and the TEM owner don't trust each other.
That is, the TEM will protect the closure's integrity and confidentiality
against its owner's attacks, and will protect itself (and the other
compiled closures of the TEM owner) against attacks from a malicious closure
author.

The TEM executes compiled closures in sequential order, in a tamper-resistant
environment. The execution environment offered by the TEM consists of a virtual
machine interpreter with a stack based instruction set, and a single flat
memory space that contains executable instructions and temporary variables. The
environment is augmented with a cryptographic engine providing standard
primitives and secure key storage, and with a persistent store designed to
guarantee the integrity and confidentiality of the variables whose values must
persist across closure executions. The persistent store is designed to use
external untrusted memory, so its capacity is not limited by the small amounts
of trusted memory available on inexpensive secure hardware.

The TEM's design focuses on offering elegance and simplicity to the software
developer (the closure author). The instruction set is small and consistent,
the memory model is easy to understand, and the persistent store has the minimal
interface of an associative memory. 

The TEM was implemented on a JavaCard smart card that uses the same family of
secure chips that are employed by Trusted Platform Module (TPM)
implementations. The TEM's prototype implementation is a living proof that the
design is practical and economical. The research code implements a full stack
of TEM software: firmware for the smart card, a Ruby extension for accessing
PC/SC smart card readers, a TEM driver, and demo software that leverages the
driver. The protoype implementation leverages the advanced features of the
Ruby language to provide a state of the art assembler which makes writing
compiled closuers for the TEM very convenient.

The Trusted Execution Module has a lot of potential to enable new applications,
by combining the flexibility of a virtual machine guaranteeing trusted
execution with the pervasiveness of inexpensive secure chips. For example, the
TEM can bring solutions to the previously unsolved problems of secure mobile
agents, secure peer-to-peer multiplayer online games, and secure anonymous
offline payments.
