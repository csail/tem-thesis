\section{TEM Features}\label{intro:features}
The Trusted Execution Module (TEM) was inspired by the Trusted Platform Module,
and it follows the principles (described in Section \ref{intro:tpm}) that led to
its widespread adoption.

The breakthrough provided by the TEM is the capability to execute user-provided
procedures in a trusted environment, for the low price of a commodity chip.
This makes the TEM capable of revolutionizing consumer software security in
the same way that the graphical quality of consumer software user interfaces
was revolutionzed by the switch from fixed-pipeline to programmable GPUs.

Most importantly, the TEM does not require any trusted software outside the
secured chip. The TPM can make a trusted statement of whether its host computer
is running trusted software or not. However, the TPM is nearly useless if its
host is not running trusted software, whereas the TEM considers this state to
be the normal operation mode.

Since the TEM does not assume trusted software on its host, it does not need to
certify the host software. Therefore, the TEM hardware does not need to be
securely bound to its host. This means that a TEM can cost less than a TPM, and
that existing computers can be enhanced with TEMs via standard extension
buses, like the USB.

Switching from fixed-function to a programmable architecture provides simpler
alternatives to the TPM's complex mechanisms. This lowers the barrier to
designing and producing software that leverages the secure module. The best
example of the simplifications achieved is replacing the TPM's hierarchical
storage scheme with a conceptually simple associative memory. Furthermore, key
migration is removed from the core architecture, as it can be achieved
completely by user procedures.

The TEM does not trust the authors of the programs it runs. A malicious
TEM program cannot negatively impact the module it runs on, and it cannot
interfere with the result of running programs written by other authors. This
feature implies there is no need for a program certification system, like
the authentication schemes used on gaming consoles. So the barriers for TEM
program developers are as low as possible.

Last but not least, the TEM can be used as a drop-in replacement for the TPM,
in the applications that don't assume trusted software on the host computer.
Indeed, it is possible to implement the TPM's functionality as user-supplied TEM
programs, while maintaining the security guarantees provided by TPM chips. This
can ease the transition from TEMs to TPMs. Note that realistic TPM applications
cannot assume trusted software on the TPM host, because there is no trusted
software stack for PC computers.
