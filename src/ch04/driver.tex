\section{Driver Software for the TEM Host}\label{impl:driver}
The prototype implementation of the TEM's driver is not at all what one would
expect to find in a driver. The low-level details of communicating with the
TEM smart card are handled by the \textsc{smartcard} Ruby extension described
in section \ref{impl:rubygem}. Ruby makes encoding and decoding APDUs trivial,
as demonstrated by listing \ref{impl:driver_alloc_buffer}, which contains the
code for allocating a buffer on the TEM. This left time and energy for higher
level features that are rarely present in production drivers, let alone
research prototypes.

\lstinputlisting[float=bph, language=Ruby, caption=TEM Buffer Allocation
Implementation in the Ruby Driver,
label=impl:driver_alloc_buffer]{code/drv_alloc_buffer.rb}

\subsection{Domain Specific Languages}
Ruby is a highly dynamic language, and it is very suited for producing DSLs
(domain-specific languages), as demonstrated in \cite{cuadrado2007bds} and
\cite{cunningham2007rmr}. DSLs are part of the ``magic'' that is responsible
for the huge success of the Ruby on Rails \cite{thomas2006awd} framework. The
TEM driver uses two DSLs directly. A small language describes the primitive
data types on a TEM, and a second-order DSL\footnote{(a DSL that generates
another DSL)} expresses all the necessary information for assembling SEClosures.

The primitive data types DSL takes types one-line type definitions and
produces functions necessary for encoding and decoding numbers corresponding to
these data types. For instance, the definition for \texttt{byte} at the
beginning of listing \ref{impl:driver_types_dsl} yields the methods
\texttt{read\_tem\_byte}, \texttt{to\_tem\_byte}, and
\texttt{tem\_byte\_length}.

\lstinputlisting[float, language=Ruby, caption=TEM Primitive Data Types
Expressed using the DSL, label=impl:driver_types_dsl]{code/drv_types_dsl.rb}

The DSL describing the instruction set of the TEM is a second-order DSL. The
instruction set DSL, illustrated in listing \ref{impl:driver_instructions_dsl},
constitutes the backbone of the TEM assembler. An instruction definition in the
DSL contains user-friendly names for the instruction and its parameters, as
well as precise instructions on how to encode the instruction for the TEM's VM.

\lstinputlisting[float, language=Ruby, caption=Memory Block Instructions
Expressed using the DSL,
label=impl:driver_instructions_dsl]{code/drv_instructions_dsl.rb}

The Ruby driver features a state of the art SECpack assembler that supports
commens, multiple instructions per line, named parameters, named labels,
human-readable immediates, macros written in Ruby, and line-level debugging
information (exploited for the advanced developer support described  in section
\ref{impl:driver_dev_support}). Relating to the bank account example introduced
in section \ref{concepts:closures}, listing \ref{impl:driver_balance_secpack}
shows a possibile implementation of the \texttt{balance} SEClosure. The power of
the assembly language is illustrated well by the SECpack source in listing
\ref{impl:driver_keygen_secpack}. Most features were obtained at very little
cost, by using a DSL to represent the assembly language. This DSL is generated
from the instruction set DSL described above.

\lstinputlisting[float=hbt, language=Ruby,
caption=Assembly code for \texttt{balance} in the bank account example,
label=impl:driver_balance_secpack]{code/drv_balance_secpack.rb}

\lstinputlisting[float=hbt, language=Ruby, caption=Assembly Code for
Key-Generating SECpack, label=impl:driver_keygen_secpack]{code/drv_keygen_secpack.rb}

The set of DSLs presented here made experimenting fun again. Once the code for
interpreting the DSLs was solid, and the initial definitions were created, it
became really easy to change the definitions to experiment variations in
encoding data types or instructions. Because the DSLs are so easy to maintain,
they have become the authoritative specifications for the TEM's encoding
mechanisms.

\subsection{Developer Support}\label{impl:driver_dev_support}
The powerful assembly language described above plays a major role in making it
easy to develop SEClosures for the TEM. The other two big features are a full
suite of unit tests, and meaningful translations for SEClosure execution errors.

The TEM driver contains a full\footnote{\texttt{rcov} indicates a line coverage
of 95\% or above on each Ruby source file.} suite of unit tests, covering both
driver and firmware code. The unit tests have proven to be a very good
investment, as they have automated the following tasks:
\begin{itemize}
  \item validating the Ruby driver implementation,
  \item validating any TEM firmware implementation, 
  \item assessing the suitability of a JavaCard model as a TEM, and
  \item assessing the compatibility of a smart card reader with the TEM driver
  stack.
\end{itemize}

When any of the layers above exhibits a bug, a failing unit test provides a good
starting point for investigation. However, if the failure is a SEClosure execution exception, the Ruby driver goes one step further, by retrieving the TEM's execution engine status (using the features in section
\ref{impl:fw_dev_support}), and combining it with line-level debugging
information. The result is the ability to pinpoint the exact SECpack
instruction that caused the failure. Listing \ref{impl:driver_exception} shows
the data obtained from a typical SEClosure execution exception. Quick inspection
reveals that the exception occured during the \texttt{outw} instruction, and the
instruction was assembled in file \texttt{test\_exceptions.rb} at line 32. The
information includes a snapshot of the execution environment at the time of the
exception.

\lstinputlisting[float, language=Ruby, caption=Debugging Information for a SEClosure Execution Exception, label=impl:driver_exception]{code/drv_exception.txt}

This level of support is hard to find in research prototypes. It is present in
the TEM's driver because Ruby's dynamic features drove the cost of implementing
the features below the time that they saved.