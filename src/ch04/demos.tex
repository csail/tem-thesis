\section{Demonstration Software}\label{impl:demos}
The TEM stack described in sections \ref{impl:firmware} - \ref{impl:driver} was
used to build the following small demonstrations:
\begin{itemize}
  \item An OpenSSL \cite{engelschall2001oos} engine using SECpacks as keys.
  When the engine is asked to generate an asymmetric key, it generates a
  2048-bit RSA key, and embeds the private key into SEClosures that encrypt,
  decrypt, and sign their arguments. The OpenSSL engine then performs all the 
  private key operations using SEClosures.
  \item A personal DRM system that uses SECpacks to store the authorization
  information for a media file. The file is broken into fixed-size blocks
  encrypted with indivual keys. The SEClosure that grants access to a file has a 
  master key used to generate the decryption keys. The SEClosure also contains
  code that expresses the permissions that the owner has (e.g., play this song at
  most 5 times, or distribute it to at most 3 friends.) The TEM provides a secure
  environment for the execution of the DRM policy code (what permissions a user
  does), and is intended to be used in conjunction with, not as a replacement
  for, secure audio playback hardware.
  \item A distributed file system that uses the TEM to insure the freshness of
  the file system's extents, which are stored on untrusted media. The servers
  use SEClosures to implement a trusted monotonic counter.
\end{itemize}

The demonstrations above show that the implementation described in this chapter
is functional and delivers acceptable performance. The chosen scenarios do not
showcase the full potential of the TEM.
